%% talk3.tex
%% Copyright 2022 Tom M. Ragonneau
%
% This work may be distributed and/or modified under the
% conditions of the LaTeX Project Public License, either version 1.3
% of this license or (at your option) any later version.
% The latest version of this license is in
%   http://www.latex-project.org/lppl.txt
% and version 1.3 or later is part of all distributions of LaTeX
% version 2005/12/01 or later.
%
% This work has the LPPL maintenance status `maintained'.
%
% The Current Maintainer of this work is Tom M. Ragonneau.
\documentclass{polyu-presentation}
\usepackage{microtype}
\usepackage{booktabs}

% List of hyphenation exceptions for US English
% Source: https://ctan.org/tex-archive/info/digests/tugboat/hyphenex
\input{ushyphex}

% Bibliographical resources
\addbibresource{ragonneau-bib/strings.bib}
\addbibresource{ragonneau-bib/optim.bib}

% Dedicated mathematical macros
\newcommand{\aub}{A}
\newcommand{\bub}{b}
\newcommand{\con}[1]{c_{#1}}
\newcommand{\conm}[2][]{\hat{c}_{#2}\ifthenelse{\equal{#1}{}}{}{^{#1}}}
\newcommand{\iub}{\mathcal{I}}
\newcommand{\obj}{f}
\newcommand{\objm}[1][]{\hat{f}\ifthenelse{\equal{#1}{}}{}{^{#1}}}
\newcommand{\xl}{l}
\newcommand{\xpt}[1][]{\mathcal{Y}\ifthenelse{\equal{#1}{}}{}{^{#1}}}
\newcommand{\xu}{u}

% Performance and data profiles
\usepackage{xstring}
\newcommand{\drawprofiles}[4]{%
    \def\selectsolvers{#2}%
    \def\selectcsv{figures/#3}%
    \def\selectprofile{#4}%
    \ifthenelse{\equal{#1}{performance}}{%
        \def\selectxlabel{$\log_2(\alpha)$}%
        \def\selectylabel{Performance profiles~$\rho_s(\alpha)$}%
    }{%
        \def\selectxlabel{Number of simplex gradients~$\alpha$}%
        \def\selectylabel{Data profiles~$d_s(\alpha)$}%
    }
    \input{figures/profiles.tex}%
}
\newcommand{\drawperformanceprofiles}[3]{\drawprofiles{performance}{#1}{#2}{#3}}
\newcommand{\drawdataprofiles}[3]{\drawprofiles{data}{#1}{#2}{#3}}

\title{Model-Based DFO Methods and Software}
\subtitle[Development of the PDFO package]{Talk no.\ 3 \textemdash\ Development of the PDFO package}
\author[Tom M. Ragonneau]{\texorpdfstring{
    Tom M. Ragonneau\\ 
    \footnotesize Co-supervised by Dr.\ Zaikun Zhang and by Prof.\ Xiaojun Chen
}{Tom M. Ragonneau}}
\institute[PolyU AMA]{
    Department of Applied Mathematics\\
    The Hong Kong Polytechnic University
}
\date{October 10, 2022}
\titlegraphic{}

\begin{document}

\begin{frame}
	\titlepage
\end{frame}

\begin{frame}
    \frametitle{Table of contents}

	\tableofcontents[hideallsubsections]
\end{frame}

\section{Introduction and motivation}

\begin{frame}
    \frametitle{Powell's DFO solvers}

	We want to solve
    \begin{equation*}
        \min_{x \in \Omega \subseteq \R^n} \obj(x),
    \end{equation*}
    where \alert{derivatives} of~$\obj$ (and the functions underlying~$\Omega$) are \alert{unknown}.

    \smallskip

    \begin{center}
        \begin{tabular}{lll}
            \toprule
            Solver  & Feasible set~$\Omega$                                 & References\\
            \midrule
            UOBYQA  & $\R^n$                                                & \cite{Powell_2002}\\
            NEWUOA  & $\R^n$                                                & \cite{Powell_2006,Powell_2008}\\
            BOBYQA  & $\set{x \in \R^n : \xl \le x \le \xu}$                & \cite{Powell_2009}\\
            LINCOA  & $\set{x \in \R^n : \aub x \le \bub}$                  & \cite{Powell_2015}\\
            COBYLA  & $\set{x \in \R^n : \con{i}(x) \ge 0, ~ i \in \iub}$   & \cite{Powell_1994}\\
            \bottomrule
        \end{tabular}
    \end{center}

    \begin{block}{An obstacle to using Powell's DFO solvers}
        Powell implemented them in \alert{Fortran 77}\dots
    \end{block}
\end{frame}

\begin{frame}
    \frametitle{The need for PDFO}

    \begin{block}{}
        \alert{PDFO} is a \alert{cross-platform} package providing \alert{MATLAB and Python} interfaces for using Powell's DFO solvers (as of v1.2).
    \end{block}

    \medskip

    \begin{enumerate}
        \item \alert{MATLAB} signature consistent with \texttt{fmincon}.
        \item \alert{Python} signature consistent with \texttt{scipy.optimize.minimize}.
        \item \alert{Open-source} and distributed under the LGPLv3+ license.
    \end{enumerate}

    \medskip
    
    \begin{columns}
        \begin{column}{0.5\textwidth}
            \begin{center}
                \includegraphics[width=1in]{images/qr-codes/pdfo.png}

                \footnotesize\url{https://www.pdfo.net/}
            \end{center}
        \end{column}
        \begin{column}{0.5\textwidth}
            \begin{center}
                \includegraphics[width=1in]{images/qr-codes/pdfo-github.png}

                \footnotesize\url{https://github.com/pdfo/pdfo}
            \end{center}
        \end{column}
    \end{columns}
\end{frame}

\section{Overview of Powell's DFO methods}

\begin{frame}
    \frametitle{Overview of Powell's DFO methods}

	They are all \alert{trust-region} methods and solve iteratively
    \begin{align*}
        \min_{x \in \R^n}   & \quad \objm[k](x)\\
        \text{s.t.}         & \quad x \in \Omega^k,\\
                            & \quad \norm{x - x^k} \le \Delta^{\! k},
    \end{align*}
    where~$\objm[k]$ is a \alert{linear or quadratic} interpolation model of~$\obj$ on~$\xpt[k] \subseteq \R^n$ and
    \begin{empheq}[left={\Omega^k = \empheqlbrace}]{alignat*=2}
        & \set{x \in \R^n : \conm[k]{i}(x) \ge 0}   && \quad \text{for COBYLA,}\\
        & \Omega                                    && \quad \text{for the others,}
    \end{empheq}
    with~$\conm[k]{i}$ being a \alert{linear} interpolation model of~$\con{i}$ on~$\xpt[k]$.
\end{frame}

\begin{frame}
    \frametitle{COBYLA~\parencite{Powell_1994}}

	COBYLA builds~$\objm[k]$ and~$\conm[k]{i}$ by \alert{linear interpolation} on~$\xpt[k]$ with
    \begin{equation*}
        \card(\xpt[k]) = n + 1.
    \end{equation*}

    \begin{block}{}
        The \alert{trust-region subproblem} is solved exactly, and if
        \begin{equation*}
            \set{x \in \R^n : \conm[k]{i}(x) \ge 0} \cap \set{x \in \R^n : \norm{x - x^k} \le \Delta^{\! k}} = \emptyset,
        \end{equation*}
        then the \alert{trial point} solves approximately
        \begin{align*}
            \min_{x \in \R^n}   & \quad \max_{i \in \iub} \, [\conm[k]{i}(x)]_{-}\\
            \text{s.t.}         & \quad \norm{x - x^k} \le \Delta^{\! k}.
        \end{align*}
    \end{block}

    \medskip

    The trial point \alert{replaces} a point in~$\xpt[k]$ to obtain~$\xpt[k + 1]$.
\end{frame}

\begin{frame}
    \frametitle{COBYLA~\parencite{Powell_1994}}

    \begin{block}{What if~$\xpt[k + 1]$ is (almost) not poised?}
        $\xpt[k + 1]$ is poised iff the \alert{volume} of the simplex engendered by~$\xpt[k + 1]$ is \alert{nonzero}.
    \end{block}

    \bigskip

    At the \alert{end} of a trust-region iteration,

    \begin{columns}
        \begin{column}{0.5\textwidth}
            \begin{center}
                \begin{tikzpicture}
                    \begin{axis}[
                        axis equal image,
                        axis lines=none,
                    ]
                        \addplot3[only marks] coordinates{(0, 0, 0) (0, 0, -1.5) (0, 2, -1.5)};
                        \addplot3[only marks,visible on=<1>] coordinates{(-1, 1, -1)};
                        \addplot3[only marks,mark options={fill=Maroon},visible on=<2->] coordinates{(-1, 1, -1)};
                        \addplot3[only marks,mark options={fill=OliveGreen},visible on=<4>] coordinates{(-2, 1, -1)};
        
                        \addplot3[no marks] coordinates{(0, 0, 0) (0, 0, -1.5)};
                        \addplot3[no marks] coordinates{(0, 0, 0) (0, 2, -1.5)};
                        \addplot3[no marks] coordinates{(0, 0, -1.5) (0, 2, -1.5)};
                        \addplot3[no marks,visible on=<1-3>] coordinates{(0, 0, 0) (-1, 1, -1)};
                        \addplot3[no marks,visible on=<1-3>] coordinates{(-1, 1, -1) (0, 0, -1.5)};
                        \addplot3[no marks,densely dotted,visible on=<1-3>] coordinates{(-1, 1, -1) (0, 2, -1.5)};
                        \addplot3[no marks,RoyalPurple,visible on=<3>] coordinates{(-3, 1, -1) (-1, 1, -1)};
                        \addplot3[no marks,densely dotted,RoyalPurple,visible on=<3>] coordinates{(-1, 1, -1) (0, 1, -1)};
                        \addplot3[no marks,RoyalPurple,visible on=<3->] coordinates{(0, 1, -1) (2, 1, -1)};
                        \addplot3[no marks,visible on=<4>] coordinates{(0, 0, 0) (-2, 1, -1)};
                        \addplot3[no marks,visible on=<4>] coordinates{(0, 0, 0) (-2, 1, -1)};
                        \addplot3[no marks,visible on=<4>] coordinates{(-2, 1, -1) (0, 0, -1.5)};
                        \addplot3[no marks,densely dotted,visible on=<4>] coordinates{(-2, 1, -1) (0, 2, -1.5)};
                        \addplot3[no marks,RoyalPurple,visible on=<4>] coordinates{(-3, 1, -1) (-2, 1, -1)};
                        \addplot3[no marks,densely dotted,RoyalPurple,visible on=<4>] coordinates{(-2, 1, -1) (0, 1, -1)};
                    \end{axis}  
                \end{tikzpicture}
            \end{center}
        \end{column}
        \begin{column}{0.5\textwidth}
            \begin{enumerate}[<+->]
                \item given a bad set~$\xpt[k + 1]$,
                \item \textcolor{Maroon}{pick a point to remove from~$\xpt[k + 1]$},
                \item \textcolor{RoyalPurple}{it is replaced with a point on the direction perpendicular to the face of~$\xpt[k + 1]$ that is to the opposite of the removed point},
                \item \textcolor{OliveGreen}{to build a better~$\xpt[k + 1]$}.
            \end{enumerate}
        \end{column}
    \end{columns}
\end{frame}

\begin{frame}
    \frametitle{UOBYQA~\parencite{Powell_2002}}

	UOBYQA builds~$\objm[k]$ by \alert{quadratic interpolation} on~$\xpt[k]$ with
    \begin{equation*}
        \card(\xpt[k]) = \frac{1}{2} (n + 1) (n + 2).
    \end{equation*}

    \begin{block}{}
        The \alert{trust-region subproblem} is approximately solved using the algorithm of~\cite{More_Sorensen_1983}.
        It is based on the \alert{crucial property} that~$x^{\ast}$ is a solution to the trust-region subproblem iff there exists~$\lambda^{\ast} \ge 0$ such that
        \begin{empheq}[left={\empheqlbrace}]{alignat*=1}
            & (\nabla^2 \objm[k] + \lambda^{\ast} I_n) (x^{\ast} - x^k) = -\nabla \objm[k](x^k),\\
            & \norm{x^{\ast} - x^k} \le \Delta^{\! k},\\
            & \lambda^{\ast} (\Delta^{\! k} - \norm{x^{\ast} - x^k}) = 0,\\
            & \nabla^2 \objm[k] \succcurlyeq -\lambda^{\ast} I_n.
        \end{empheq}
    \end{block}

    \medskip

    The trial point \alert{replaces} a point in~$\xpt[k]$ to obtain~$\xpt[k + 1]$.
\end{frame}

\begin{frame}
    \frametitle{UOBYQA~\parencite{Powell_2002}}

    \begin{block}{What if~$\xpt[k + 1]$ is (almost) not poised?}
        Let B be the coefficient matrix of the interpolation system on~$\xpt[k + 1]$, and let~$B_y(x)$ be be that on~$(\xpt[k + 1] \setminus \set{y}) \cup \set{x}$ with~$y \in \xpt[k + 1]$.
        \cite{Powell_2001} shows that the \alert{Lagrange polynomial} for~$y$ satisfies
        \begin{equation*}
            L_y(x) = \frac{\det B_y(x)}{\det B}.
        \end{equation*}
        % An analysis in \cite{Powell_2001} shows that replacing a point~$y \in \xpt[k + 1]$ by another one that provides a large value of~$\abs{L_y(\cdot)}$ is
    \end{block}

    \medskip

    A point~$y \in \xpt[k + 1]$ is \alert{replaced} with an approximate solution to
    \begin{align*}
        \max_{x \in \R^n}   & \quad \abs{L_y(x)}\\
        \text{s.t.}         & \quad \norm{x - x^{k + 1}} \le \bar{\Delta}^{\! k},
    \end{align*}
    for some~$\bar{\Delta}^{\! k} \ge 0$.
\end{frame}

\begin{frame}
    \frametitle{NEWUOA, BOBYQA, LINCOA~\parencite{Powell_2006,Powell_2009,Powell_2015}}

	They build~$\objm[k]$ by \alert{underdetermined quadratic interpolation} on~$\xpt[k]$ with
    \begin{equation*}
        n + 2 \le \card(\xpt[k]) \le \frac{1}{2} (n + 1) (n + 2).
    \end{equation*}
    They employ the \alert{derivative-free symmetric Broyden update}
    \begin{align*}
        \min        & \quad \frac{1}{2} \norm{\nabla^2 Q - \nabla^2 \objm[k - 1]}_{\mathsf{F}}^2\\
        \text{s.t.} & \quad Q(y) = f(y), ~ y \in \xpt[k].
    \end{align*}

    \begin{block}{}
        The \alert{trust-region subproblems} are approximately solved using the \alert{truncated conjugate gradient} algorithm of~\cite{Steihaug_1983} and~\cite{Toint_1981}.
    \end{block}

    \medskip

    The trial point \alert{replaces} a point in~$\xpt[k]$ to obtain~$\xpt[k + 1]$.
\end{frame}

\begin{frame}
    \frametitle{NEWUOA, BOBYQA, LINCOA~\parencite{Powell_2006,Powell_2009,Powell_2015}}

    \begin{block}{What if~$\xpt[k + 1]$ is (almost) not poised?}
        To update the \alert{inverse} of the coefficient matrix of the interpolation system from~$\xpt[k + 1]$ to~$\xpt[k]$, they use the \alert{Woodbury formula}.
        \cite{Powell_2004c} shows that the only denominator of this update is bounded from below by
        \begin{equation*}
            L_y(x)^2, \quad \text{with~$\xpt[k + 1] = (\xpt[k] \setminus \set{y}) \cup \set{x}$},
        \end{equation*}
        where~$L_y$ is the \alert{Lagrange polynomial} for~$y$ in the least Frobenius norm sense.
    \end{block}

    \smallskip

    A point~$y \in \xpt[k + 1]$ is \alert{replaced} with an approximate solution to
    \begin{align*}
        \max_{x \in \Omega} & \quad \abs{L_y(x)}\\
        \text{s.t.}         & \quad \norm{x - x^{k + 1}} \le \bar{\Delta}^{\! k},
    \end{align*}
    for some~$\bar{\Delta}^{\! k} \ge 0$.
\end{frame}

\section{Core features of the PDFO package}

\begin{frame}[fragile]
    \frametitle{Solver selection}

    \begin{block}{How to call PDFO (e.g., in MATLAB)?}
        \begin{lstmatlab}
            x = pdfo(@fun, x0, A, b, Aeq, beq, lb, ub, @nonlcon);
        
            function fx = fun(x)
            ...
            end
        
            function [c, ceq] = nonlcon(x)
            ...
            end
        \end{lstmatlab}
    \end{block}

    \medskip

    PDFO \alert{automatically selects} the \enquote{best} solver to solve the problem.
\end{frame}

\begin{frame}
    \frametitle{Problem preprocessing}

	\begin{enumerate}
        \item \alert{Projection of~$x^0$}.
        For bound- and linearly constrained problems, PDFO projects~$x^0$ onto~$\Omega$ (BOBYQA and LINCOA would otherwise fail).
        \item \alert{Removal of linear equality constraints}.
        Assume that~$A$ has full column rank and that~$A x^0 = b$.
        Let~$A^{\T} = \hat{Q} R$ be the \alert{QR factorization} of~$A^{\T}$.
        Then
        \begin{equation*}
            A x = b \Longleftrightarrow \hat{Q}^{\T} (x - x^0) = 0, \quad \text{for~$x \in \R^n$}.
        \end{equation*}
        Let~$\check{Q}$ be such that~$\begin{bmatrix} \hat{Q} & \check{Q} \end{bmatrix}$ is \alert{orthogonal}.
        Then
        \begin{equation*}
            \set{x \in \R^n : A x = b} = \set{x^0} + \check{Q} \R^{n - \rank(A)}
        \end{equation*}
    \end{enumerate}

    \medskip

    \begin{block}{}
        This is done by PDFO, the user do \alert{not need} to do it.
    \end{block}
\end{frame}

\begin{frame}
    \frametitle{Other features of the PDFO package}

    \begin{enumerate}
        \item \alert{Comprehensiveness}.
        PDFO is the only package providing all five solvers.
        \item \alert{Code patching}.
        We patched infinite cycling, segmentation faults, \dots
        \item \alert{Fault tolerance}.
        PDFO handles faults in the problem's functions.
        \item \alert{Additional options}. We implemented additional features, e.g.,
        \begin{enumerate}
            \item scaling the problem according to the bound constraints,
            \item stopping the computations if a target value is reached.
        \end{enumerate}
    \end{enumerate}

    \medskip

    \begin{center}
        \includegraphics[width=1in]{images/qr-codes/pdfo-usage.png}

        \footnotesize\url{https://www.pdfo.net/docs.html\#usage}
    \end{center}
\end{frame}

\section{Numerical experiments}

\begin{frame}
    \frametitle{Comparisons on unconstrained problems}

    \begin{columns}
        \begin{column}{0.4\textwidth}
            We compare the solvers on
            \begin{enumerate}
                \item \alert{unconstrained} problems,
                \item of the \alert{CUTEst} data,
                \item of dimension \alert{at most \only<1>{10}\only<2->{50}},
                \item with a tolerance~$\tau = 10^{-\only<1,2>{4}\only<3>{1}\only<4>{2}}$\only<1,2>{.}\only<3->{,}
                \uncover<3->{
                    \item replacing~$\obj$ with
                    \begin{equation*}
                        \tilde{\obj}(x) = [1 + \epsilon(x)] \obj(x),
                    \end{equation*}
                    where~$\epsilon(x) \sim N(0, 10^{-4})$.
                }
            \end{enumerate}
        \end{column}
        \begin{column}{0.6\textwidth}
            \only<1>{
                \begin{center}
                    \drawperformanceprofiles{{"NEWUOA","BOBYQA","LINCOA","COBYLA","UOBYQA"}}{plain-1-10-perf-newuoa-bobyqa-lincoa-cobyla-uobyqa-u.csv}{4}
                \end{center}
            }
            \only<2>{
                \begin{center}
                    \drawperformanceprofiles{{"NEWUOA","BOBYQA","LINCOA","COBYLA"}}{plain-1-50-perf-newuoa-bobyqa-lincoa-cobyla-u.csv}{4}
                \end{center}
            }
            \only<3>{
                \begin{center}
                    \drawperformanceprofiles{{"NEWUOA","BOBYQA","LINCOA","COBYLA"}}{noisy-1-50-2-perf-newuoa-bobyqa-lincoa-cobyla-u.csv}{1}
                \end{center}
            }
            \only<4>{
                \begin{center}
                    \drawperformanceprofiles{{"NEWUOA","BOBYQA","LINCOA","COBYLA"}}{noisy-1-50-2-perf-newuoa-bobyqa-lincoa-cobyla-u.csv}{2}
                \end{center}
            }
        \end{column}
    \end{columns}
\end{frame}

\section{Conclusion}

\begin{frame}
    \frametitle{Conclusion}
    
    In this talk, we presented
    \begin{enumerate}
        \item \alert{Powell's DFO methods},
        \item the \alert{PDFO} package for MATLAB and Python, and
        \item some \alert{numerical experiments} using PDFO.
    \end{enumerate}

    \bigskip

    The PDFO package
    \begin{enumerate}
        \item has been \alert{downloaded} more than \num{36000} times (as of October 2022),
        \item received \alert{favorable feedback} from IRT Saint-Exup{\'{e}}ry in France.
    \end{enumerate}

    \bigskip

    In the next talk, we will discuss the \alert{SQP method} in optimization.
\end{frame}

\appendix

\begin{frame}
    \frametitle{Exercises from previous group seminars}

    \begin{block}{Exercise no.\ 1}
        Is it possible to \alert{reformulate} any \alert{piecewise linear programming} problem as a \alert{linear programming} problem?
    \end{block}

    \smallskip

    \begin{enumerate}
        \item If the problem is \alert{convex}, this is \alert{true} because any convex piecewise linear function~$\obj$ can be formulated as
        \begin{equation*}
            \obj(x) = \max_{1 \le i \le m} a_i^{\T} x + b_i.
        \end{equation*}
        \item If the problem is \alert{nonconvex}, this is \alert{not true}.
        Consider
        \begin{equation*}
            \min_{x \in [-1, 1]} -\abs{x},
        \end{equation*}
        whose solutions are~$\pm 1$.
        If this problem could be reformulated as a linear programming problem, any~$x \in [-1, 1]$ would be a solution.
    \end{enumerate}
\end{frame}

\begin{frame}
    \frametitle{Exercises from previous group seminars}

    \begin{block}{Exercise no.\ 2}
        Reformulate
        \begin{align*}
            \min_{x, g, H}  & \quad \norm{H}\\
            \text{s.t.}     & \quad c + g^{\T} x + \frac{1}{2} x^{\T} H x = \obj(y), ~ y \in \xpt
        \end{align*}
        as \alert{linear programming} problems when~$\norm{\cdot}$ is the~$\ell_1$- or the~$\ell_{\infty}$-induced norm.
    \end{block}

    \bigskip

    \begin{enumerate}
        \item Since~$H$ is \alert{symmetric},~$\norm{H}_1 = \norm{H}_{\infty}$.
        The reformulations are \alert{identical}.
        \item We use the fact that
        \begin{equation*}
            \norm{H} = \max_{1 \le j \le n} \sum_{i = 1}^n \abs{H_{i, j}} = \max_{1 \le j \le n} \sum_{i = 1}^n \max \set{H_{i, j}, -H_{i, j}}.
        \end{equation*}
    \end{enumerate}
\end{frame}

\begin{frame}
    \frametitle{Exercises from previous group seminars}

    Therefore, a \alert{reformulated problem} is
        \begin{align*}
            \min_{x, g, H, U, t}    & \quad t\\
            \text{s.t.}             & \quad \sum_{i = 1}^n U_{i, j} \le t, ~ j \in \set{1, 2, \dots, n},\\
                                    & \quad H \le U,\\
                                    & \quad -H \le U,\\
                                    & \quad c + g^{\T} x + \frac{1}{2} x^{\T} H x = \obj(y), ~ y \in \xpt.
        \end{align*}
\end{frame}

\begin{frame}
    \frametitle{Exercises from previous group seminars}

    \begin{block}{Exercise no.\ 3}
        Is the \alert{error estimation} in~\cite{Conn_Scheinberg_Vicente_2009b} on polynomial interpolation correct?
        Is it \alert{consistent} with the theory in~\cite{Ciarlet_Raviart_1972}?
    \end{block}

    Let~$\xpt \subseteq \R^n$ be a \alert{poised} interpolation set in~$\R_k[X^n]$ and let~$u$ be a (regular enough) function.
    For all~$m$ with~$0 \le m \le k$, the interpolant~$\hat{u}$ of~$u$ satisfies
    \begin{equation*}
        \sup_{x \in \conv(\xpt)} \norm{\nabla^m \hat{u}(x) - \nabla^m u(x)} \le \frac{\nu \diam(\xpt)^{m + k + 1}}{\rho^m (k + 1)!} \sum_{y \in \xpt} \sup_{x \in \conv(\xpt)} \norm{\nabla^m L_y(x)},
    \end{equation*}
    where~$L_y$ is the \alert{Lagrange polynomial} for~$y \in \xpt$,~$\rho$ is the diameter of the largest~$\ell_2$-inscribed sphere of~$\conv(\xpt)$, and
    \begin{equation*}
        \nu = \sup_{x \in \conv(\xpt)} \norm{\nabla^{k + 1} u(x)},
    \end{equation*}
    with~$\norm{\cdot}$ being the~$\ell_2$-induced norm.
\end{frame}

\begin{frame}
    \frametitle{Exercises from previous group seminars}

    However, the bound in~\cite{Conn_Scheinberg_Vicente_2009b} leads to
    \begin{equation*}
        \sup_{x \in \conv(\xpt)} \norm{\nabla^m \hat{u}(x) - \nabla^m u(x)} \le \frac{\nu \diam(\xpt)^{k + 1}}{(k + 1)!} \sum_{y \in \xpt} \sup_{x \in \conv(\xpt)} \norm{\nabla^m L_y(x)}.
    \end{equation*}
    This \alert{cannot be implied} by~\cite{Ciarlet_Raviart_1972}, because~\alert{$\diam(\xpt) > \rho$}.

    \bigskip

    \begin{columns}
        \begin{column}{0.4\textwidth}
            \begin{center}
                \begin{tikzpicture}
                    % Coordinates of the vertices of the triangle
                    \coordinate[label=below left:$y^1$] (a) at (0,0);
                    \coordinate[label=above right:$y^2$] (b) at (1.5,2);
                    \coordinate[label=below right:$y^3$] (c) at (2,0.5);
        
                    % Triangle
                    \draw[thick] (a) -- (b) -- (c) -- cycle;
        
                    % Circumscribed circle
                    \draw[thick,OliveGreen] let
                        \p1=(a),
                        \p2=(b),
                        \p3=(c),
                        \n1={veclen(\x1-\x2,\y1-\y2)},
                        \n2={veclen(\x2-\x3,\y2-\y3)},
                        \n3={veclen(\x3-\x1,\y3-\y1)},
                        \n4={acos((\n1^2+\n3^2-\n2^2)/(2*\n1*\n3))},
                        \n5={acos((\n1^2+\n2^2-\n3^2)/(2*\n1*\n2))},
                        \n6={acos((\n2^2+\n3^2-\n1^2)/(2*\n2*\n3))},
                        \n7={0.5*(\n1+\n2+\n3)},
                    in
                        ({(\x1*sin(2*\n4)+\x2*sin(2*\n5)+\x3*sin(2*\n6))/(sin(2*\n4)+sin(2*\n5)+sin(2*\n6))},{(\y1*sin(2*\n4)+\y2*sin(2*\n5)+\y3*sin(2*\n6))/(sin(2*\n4)+sin(2*\n5)+sin(2*\n6))}) circle[radius={0.25*(1/sqrt(\n7))*(1/sqrt((\n7-\n1)/(\n1^2)))*(1/sqrt((\n7-\n2)/(\n2^2)))*(1/sqrt((\n7-\n3)/(\n3^2)))}];
        
                    % Inscribed circle
                    \draw[thick,Maroon] let
                        \p1=(a),
                        \p2=(b),
                        \p3=(c),
                        \n1={veclen(\x1-\x2,\y1-\y2)},
                        \n2={veclen(\x2-\x3,\y2-\y3)},
                        \n3={veclen(\x3-\x1,\y3-\y1)},
                        \n4={0.5*(\n1+\n2+\n3)},
                    in
                        (({(\x1*\n2+\x2*\n3+\x3*\n1)/(\n1+\n2+\n3)},{(\y1*\n2+\y2*\n3+\y3*\n1)/(\n1+\n2+\n3)}) circle[radius=(1/sqrt(\n4))*sqrt(\n4-\n1)*sqrt(\n4-\n2)*sqrt(\n4-\n3)];
                    
                    % Points
                    \foreach \x in {(a),(b),(c)}{
                        \fill \x circle[radius=2pt];
                    }
                \end{tikzpicture}
            \end{center}
        \end{column}
        \begin{column}{0.6\textwidth}
            With~$\xpt = \set{y^1, y^2, y^3} \subseteq \R^2$, the diameter of the \textcolor{Maroon}{inscribed circle} is smaller than the diameter of the \textcolor{OliveGreen}{circumscribed circle}, itself smaller than~$\diam(\xpt)$.
        \end{column}
    \end{columns}
\end{frame}

\begin{frame}[t,allowframebreaks]
    \frametitle{References}

	\printbibliography
\end{frame}

\end{document}